\newpage
\section{Prolog directives and control constructs}
%HEVEA\cutdef[1]{subsection}
\subsection{Prolog directives}

\subsubsection{Introduction}
Prolog directives are annotations inserted in Prolog source files for the
compiler. A Prolog directive is used to specify:

\begin{itemize}

\item the properties of some procedures defined in the source file.

\item the format and the syntax for read-terms in the source file (using
changeable Prolog flags).

\item included source files.

\item a goal to be executed at run-time.

\end{itemize}

\subsubsection{\IdxDiD{dynamic/1} \label{dynamic/1}}

\begin{TemplatesOneCol}
dynamic(+predicate\_indicator)\\
dynamic(+predicate\_indicator\_list)\\
dynamic(+predicate\_indicator\_sequence)

\end{TemplatesOneCol}

\Description

\texttt{dynamic(Pred)} specifies that the procedure whose
predicate indicator is \texttt{Pred} is a dynamic procedure. This directive
makes it possible to alter the definition of \texttt{Pred} by adding or
removing clauses. For more information refer to the section about dynamic
clause management \RefSP{Introduction:(Dynamic-clause-management)}.

This directive shall precede the definition of \texttt{Pred} in the source
file.

If there is no clause for \texttt{Pred} in the source file, \texttt{Pred}
exists however as an empty predicate (this means that
\texttt{current\_predicate(Pred)} succeeds).

In order to allow multiple definitions, \texttt{Pred} can also be a list of
predicate indicators or a sequence of predicate indicators using
\texttt{','/2} as separator.

\Portability

ISO directive.

\subsubsection{\IdxDiD{public/1} \label{public/1}}

\begin{TemplatesOneCol}
public(+predicate\_indicator)\\
public(+predicate\_indicator\_list)\\
public(+predicate\_indicator\_sequence)

\end{TemplatesOneCol}

\Description

\texttt{public(Pred)} specifies that the procedure whose predicate indicator
is \texttt{Pred} is a public procedure. This directive makes it possible to
inspect the clauses of \texttt{Pred}. For more information refer to the
section about dynamic clause management \RefSP{Introduction:(Dynamic-clause-management)}.

This directive shall precede the definition of \texttt{Pred} in the source
file. Since a dynamic procedure is also public. It is useless (but correct)
to define a public directive for a predicate already declared as dynamic.

In order to allow multiple definitions, \texttt{Pred} can also be a list of
predicate indicators or a sequence of predicate indicators using
\texttt{','/2} as separator.

\Portability

GNU Prolog directive. The ISO reference does not define any directive to
declare a predicate public but it does distinguish public predicates. It is
worth noting that in most Prolog systems the \texttt{public/1} directive is
as a visibility declaration. Indeed, declaring a predicate as public makes
it visible from any predicate defined in any other file (otherwise the
predicate is only visible from predicates defined in the same source file as
itself). When a module system is incorporated in GNU Prolog a more general
visibility declaration shall be provided conforming to the ISO reference.

\subsubsection{\IdxDiD{multifile/1}}

\begin{TemplatesOneCol}
multifile(+predicate\_indicator)\\
multifile(+predicate\_indicator\_list)\\
multifile(+predicate\_indicator\_sequence)

\end{TemplatesOneCol}

\Description

\texttt{multifile(Pred)} specifies that the procedure whose predicate
indicator is \texttt{Pred} is a multifle procedure (the clauses of
\texttt{Pred} can reside in several source files). This directive is only
supported by GNU Prolog since version 1.4.0.

\Portability

ISO directive.

\subsubsection{\IdxDiD{discontiguous/1}}

\begin{TemplatesOneCol}
discontiguous(+predicate\_indicator)\\
discontiguous(+predicate\_indicator\_list)\\
discontiguous(+predicate\_indicator\_sequence)

\end{TemplatesOneCol}

\Description

\texttt{discontiguous(Pred)} specifies that the procedure whose predicate
indicator is \texttt{Pred} is a discontiguous procedure. Namely, the clauses
defining \texttt{Pred} are not restricted to be consecutive but can appear
anywhere in the source file.

This directive shall precede the definition of \texttt{Pred} in the source
file.

In order to allow multiple definitions, \texttt{Pred} can also be a list of
predicate indicators or a sequence of predicate indicators using
\texttt{','/2} as separator.

A multifile predicate cannot be directly called from a file where it is not
declared as multifile (the native compiler must know the called predicate is
multifile). Workarounds: either call it via a meta-call (e.g. using
\texttt{call/1}) or declare it as multifile in the calling source file). A
good habit is to encapsulate a multifile predicate in a monofile predicate
which invokes it (external call only invoke the monofile wrapper predicate).

\Portability

ISO directive. 

\subsubsection{\IdxDiD{ensure\_linked/1}}

\begin{TemplatesOneCol}
ensure\_linked(+predicate\_indicator)\\
ensure\_linked(+predicate\_indicator\_list)\\
ensure\_linked(+predicate\_indicator\_sequence)

\end{TemplatesOneCol}

\Description

\texttt{ensure\_linked(Pred)} specifies that the procedure
whose predicate indicator is \texttt{Pred} must be included by the linker.
This directive is useful when compiling to native code to force the linker to
include the code of a given predicate. Indeed, if the \texttt{gplc} is
invoked with an option to reduce the size of the executable
\RefSP{Using-the-compiler}, the linker only includes the code of predicates
that are statically referenced. However, the linker cannot detect dynamically
referenced predicates (used as data passed to a meta-call predicate). The use
of this directive prevents it to exclude the code of such predicates.

In order to allow multiple definitions, \texttt{Pred} can also be a list of
predicate indicators or a sequence of predicate indicators using
\texttt{','/2} as separator.

\Portability

GNU Prolog directive.

\subsubsection{\IdxDiD{built\_in/0},
               \IdxDiD{built\_in/1},
               \IdxDiD{built\_in\_fd/0},
               \IdxDiD{built\_in\_fd/1}}

\begin{TemplatesOneCol}
built\_in\\
built\_in(+predicate\_indicator)\\
built\_in(+predicate\_indicator\_list)\\
built\_in(+predicate\_indicator\_sequence)\\
built\_in\_fd\\
built\_in\_fd(+predicate\_indicator)\\
built\_in\_fd(+predicate\_indicator\_list)\\
built\_in\_fd(+predicate\_indicator\_sequence)

\end{TemplatesOneCol}

\Description

\texttt{built\_in} specifies that the procedures defined from
now have the \IdxPP{built\_in} property \RefSP{predicate-property/2}.

\texttt{built\_in(Pred)} is similar to \texttt{built\_in/0} but
only affects the procedure whose predicate indicator is \texttt{Pred}.

This directive shall precede the definition of \texttt{Pred} in the source
file.

In order to allow multiple definitions, \texttt{Pred} can also be a list of
predicate indicators or a sequence of predicate indicators using
\texttt{','/2} as separator.

\texttt{built\_in\_fd} (resp.
\texttt{built\_in\_fd(Pred)}) is similar to
\texttt{built\_in} (resp. \texttt{built\_in(Pred)}) but sets the
\IdxPP{built\_in\_fd} predicate property \RefSP{predicate-property/2}.

\Portability

GNU Prolog directives.

\subsubsection{\IdxDiD{include/1}}

\begin{TemplatesOneCol}
include(+atom)

\end{TemplatesOneCol}

\Description

\texttt{include(File)} specifies that the content of the Prolog source
\texttt{File} shall be inserted. The resulting Prolog text is identical to
the Prolog text obtained by replacing the directive by the content of the
Prolog source \texttt{File}.

In case of \texttt{File} is a relative file name, it is searched in the
current directory. If it is not found it is then searched in each directory
of parent includers.


See \IdxPB{absolute\_file\_name/2} for information about the syntax of
\texttt{File} \RefSP{absolute-file-name/2}.

\Portability

ISO directive.

\subsubsection{\IdxDiD{if/1}, \IdxDiD{else/0}, \IdxDiD{endif/0}, \IdxDiD{elif/1} }

\begin{TemplatesOneCol}
if(+callable\_term) \\
else\\
endif\\
elif(+callable\_term)

\end{TemplatesOneCol}

\Description

These directives are for conditional compilation.

\texttt{if(Goal)} compile subsequent code only if \texttt{Goal}
succeeds. \texttt{Goal} is first processed by \texttt{expand\_term/2}
\RefSP{expand-term/2}.  If \texttt{Goal} raises an exception it is printed
and \texttt{Goal} fails.

\texttt{else} introduces the \textit{else} part.

\texttt{endif} terminates a conditional compilation part.

\texttt{elif(Goal)} is a shorthand for \texttt{:- else. :- if(Goal). $\ldots$ :- endif}.

\Portability

GNU Prolog directive. Also in SWI and YAP.


\subsubsection{\IdxDiD{ensure\_loaded/1}}

\begin{TemplatesOneCol}
ensure\_loaded(+atom)

\end{TemplatesOneCol}

\Description

\texttt{ensure\_loaded(File)} is not supported by GNU Prolog. When such a
directive is encountered it is simply ignored.

\Portability

ISO directive. Not supported.

\subsubsection{\IdxDiD{op/3} \label{op/3}}

\begin{TemplatesOneCol}
op(+integer, +operator\_specifier, +atom\_or\_atom\_list)

\end{TemplatesOneCol}

\Description

\texttt{op(Priority, OpSpecifier, Operator)} alters the operator table. This
directive is executed as soon as it is encountered by calling the built-in
predicate \IdxPB{op/3} \RefSP{op/3:(Term-input/output)}. A system
directive is also generated to reflect the effect of this directive at
run-time \RefSP{Running-an-executable}.

\Portability

ISO directive.

\subsubsection{\IdxDiD{char\_conversion/2}}

\begin{TemplatesOneCol}
char\_conversion(+character, +character)

\end{TemplatesOneCol}

\Description

\texttt{char\_conversion(InChar, OutChar)} alters the character-conversion
mapping. This directive is executed as soon as it is encountered by a call
to the built-in predicate \IdxPB{char\_conversion/2}
\RefSP{char-conversion/2}. A system directive is also generated to reflect
the effect of this directive at run-time \RefSP{Running-an-executable}.

\Portability

ISO directive.

\subsubsection{\IdxDiD{set\_prolog\_flag/2}}

\begin{TemplatesOneCol}
set\_prolog\_flag(+flag, +term)

\end{TemplatesOneCol}

\Description

\texttt{set\_prolog\_flag(Flag, Value)} sets the value of the
\Idx{Prolog flag} \texttt{Flag} to \texttt{Value}. This directive is
executed as soon as it is encountered by a call to the built-in predicate
\IdxPB{set\_prolog\_flag/2} \RefSP{set-prolog-flag/2}. A system directive
is also generated to reflect the effect of this directive at run-time
\RefSP{Running-an-executable}.

\Portability

ISO directive.

\subsubsection{\IdxDiD{initialization/1} \label{initialization/1}}

\begin{TemplatesOneCol}
initialization(+callable\_term)

\end{TemplatesOneCol}

\Description

\texttt{initialization(Goal)} adds \texttt{Goal} to the set of goal which
shall be executed at run-time. A user directive is generated to execute
\texttt{Goal} at run-time. If several initialization directives appear in
the same file they are executed in the order of appearance
\RefSP{Running-an-executable}.

\Portability

ISO directive.

\subsubsection{\IdxDiD{foreign/2},
               \IdxDiD{foreign/1} \label{foreign/2}}

\begin{TemplatesOneCol}
foreign(+callable\_term, +foreign\_option\_list)\\
foreign(+callable\_term)

\end{TemplatesOneCol}

\Description

\texttt{foreign(Template, Options)} defines an interface predicate whose
prototype is \texttt{Template} according to the options given by
\texttt{Options}. Refer to the foreign code interface for more information
\RefSP{Calling-C-from-Prolog}.

\texttt{foreign(Template)} is equivalent to \texttt{foreign(Template, [])}.

\Portability

GNU Prolog directive.

\subsection{Prolog control constructs}
\label{control-construct}

GNU Prolog follows the ISO notion of \Idx{control constructs}. 

\subsubsection{\IdxCCD{true/0},
               \IdxCCD{fail/0},
               \AddCCD{"!/0}\texttt{!/0} \label{true/0}} % Pb with ! in \index with HeVeA

\begin{TemplatesOneCol}
true\\
fail\\
!

\end{TemplatesOneCol}

\Description

\texttt{true} always succeeds.

\texttt{fail} always fails (enforces backtracking).

\texttt{!} always succeeds and the for side-effect of removing all
choice-points created since the invocation of the predicate activating it.

\PlErrorsNone

\Portability

ISO control constructs.

\subsubsection{\IdxCCD{(',')/2} - conjunction,
               \IdxCCD{(;)/2} - disjunction,
               \IdxCCD{(-{\gt})/2} - if-then, 
               \IdxCCD{(*-{\gt})/2} - soft-cut (soft if-then)}

\begin{TemplatesOneCol}
','(+callable\_term, +callable\_term)\\
;(+callable\_term, +callable\_term)\\
-{\gt}(+callable\_term, +callable\_term)\\
{*-{\gt}}(+callable\_term, +callable\_term)

\end{TemplatesOneCol}

\Description

\texttt{Goal1 , Goal2} executes \texttt{Goal1} and, in case of
success, executes \texttt{Goal2}.

\texttt{Goal1 ; Goal2} first creates a choice-point and executes
\texttt{Goal1}. On backtracking \texttt{Goal2} is executed.

\texttt{Goal1 -{\gt} Goal2} first executes \texttt{Goal1} and,
in case of success, removes all choice-points created by \texttt{Goal1} and
executes \texttt{Goal2}. This control construct acts like an if-then
(\texttt{Goal1} is the test part and \texttt{Goal2} the then part). Note that
if \texttt{Goal1} fails \texttt{-{\gt}/2} fails also. \texttt{-{\gt}/2} is
often combined with \texttt{;/2} to define an if-then-else as follows:
\texttt{Goal1 -{\gt} Goal2 ; Goal3}. Note that \texttt{Goal1 -{\gt} Goal2}
is the first argument of the \texttt{(;)/2} and \texttt{Goal3} (the else
part) is the second argument. Such an if-then-else control construct first
creates a choice-point for the else-part (intuitively associated with
\texttt{;/2}) and then executes \texttt{Goal1}. In case of success, all
choice-points created by \texttt{Goal1} together with the choice-point for
the else-part are removed and \texttt{Goal2} is executed. If \texttt{Goal1}
fails then \texttt{Goal3} is executed.

\texttt{Goal1 *-{\gt} Goal2 ; Goal3} implements the so-called
\Idx{soft-cut}. It acts as the above if-then-else except that if
\texttt{Goal1} succeeds only \texttt{Goal3} is cut (the alternative solutions
of \texttt{Goal1} are preserved and can be found by backtracking). Note that
\texttt{Goal1 *-{\gt} Goal2} alone (i.e. without an else branch
\texttt{Goal3}) is equivalent to \texttt{(Goal1 , Goal2)}.


\texttt{','}, \texttt{;}, \texttt{-{\gt}} and \texttt{*-{\gt}} are predefined
infix operators \RefSP{op/3:(Term-input/output)}.

\begin{PlErrors}

\ErrCond{\texttt{Goal1} or \texttt{Goal2} is a variable}
\ErrTerm{instantiation\_error}

\ErrCond{\texttt{Goal1} is neither a variable nor a callable term}
\ErrTerm{type\_error(callable, Goal1)}

\ErrCond{\texttt{Goal2} is neither a variable nor a callable term}
\ErrTerm{type\_error(callable, Goal2)}

\ErrCond{The predicate indicator \texttt{Pred} of \texttt{Goal1} or
\texttt{Goal2} does not correspond to an existing procedure and the value of
the \texttt{unknown} Prolog flag is \texttt{error} \RefSP{set-prolog-flag/2}}
\ErrTerm{existence\_error(procedure, Pred)}

\end{PlErrors}

\Portability

ISO control constructs except \texttt{(*-{\gt})/2} which is GNU Prolog specific.

\subsubsection{\IdxCCD{call/1} \label{call/1}}

\begin{TemplatesOneCol}
call(+callable\_term)

\end{TemplatesOneCol}

\Description

\texttt{call(Goal)} executes \texttt{Goal}. \texttt{call/1} succeeds if
\texttt{Goal} represents a goal which is true. When \texttt{Goal} contains a
cut symbol \AddCC{"!/0}\texttt{!} \RefSP{true/0} as a subgoal, the effect of
\texttt{!} does not extend outside \texttt{Goal}.

\begin{PlErrors}

\ErrCond{\texttt{Goal} is a variable}
\ErrTerm{instantiation\_error}

\ErrCond{\texttt{Goal} is neither a variable nor a callable term}
\ErrTerm{type\_error(callable, Goal)}

\ErrCond{The predicate indicator \texttt{Pred} of \texttt{Goal} does not
correspond to an existing procedure and the value of the \texttt{unknown}
Prolog flag is \texttt{error} \RefSP{set-prolog-flag/2}}
\ErrTerm{existence\_error(procedure, Pred)}

\end{PlErrors}

\Portability

ISO control construct.

\subsubsection{\IdxCCD{catch/3},
               \IdxCCD{throw/1} \label{catch/3}}

\begin{TemplatesOneCol}
catch(?callable\_term, ?term, ?term)\\
throw(+nonvar)

\end{TemplatesOneCol}

\Description

\texttt{catch(Goal, Catcher, Recovery)} is similar to \texttt{call(Goal)}
\RefSP{call/1}. If this succeeds or fails, so does the call to
\texttt{catch/3}. If however, during the execution of \texttt{Goal}, there
is a call to \texttt{throw(Ball)}, the current flow of control is
interrupted, and control returns to a call of \texttt{catch/3} that is being
executed. This can happen in one of two ways:

\begin{itemize}

\item implicitly, when an error condition for a built-in predicate is
satisfied.

\item explicitly, when the program executes a call of \texttt{throw/1}
because the program wishes to abandon the current processing, and instead to
take an alternative action.

\end{itemize}

\texttt{throw(Ball)} causes the normal flow of control to be transferred
back to an existing call of \texttt{catch/3}. When a call to
\texttt{throw(Ball)} happens, \texttt{Ball} is copied and the stack is
unwound back to the call to \texttt{catch/3}, whereupon the copy of
\texttt{Ball} is unified with \texttt{Catcher}. If this unification
succeeds, then \texttt{catch/3} executes the goal \texttt{Recovery} using
\texttt{call/1} \RefSP{call/1} in order to determine the success or
failure of \texttt{catch/3}. Otherwise, in case the unification fails,
the stack keeps unwinding, looking for an earlier invocation of
\texttt{catch/3}. \texttt{Ball} may be any non-variable term.

\begin{PlErrors}

\ErrCond{\texttt{Ball} is a variable}
\ErrTerm{instantiation\_error}

\end{PlErrors}

If \texttt{Ball} does not unify with the \texttt{Catcher} argument of
any call of \texttt{catch/3}, a system error message is displayed and
\texttt{throw/1} fails.

When \texttt{catch/3} calls \texttt{Goal} or \texttt{Recovery} it uses \texttt{call/1}
\RefSP{call/1}, an \texttt{instantiation\_error}, a \texttt{type\_error}
or an \texttt{existence\_error} can then occur depending on
\texttt{Goal} or \texttt{Recovery}.

\Portability

ISO control constructs.

%HEVEA\cutend

